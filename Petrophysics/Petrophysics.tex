%-*-coding:UTF-8-*-
%文件名.tex
%内容主旨
\documentclass[hyperref,UTF-8,twoside]{ctexart}
\newcommand{\cndash}{\rule{0.2em}{0pt}\rule[0.35em]{1.6em}{0.05em}\rule{0.2em}{0pt}}%中文破折号命令:\cndash{}
\hypersetup{colorlinks=false,pdfborder=000}
\usepackage{geometry}
\renewcommand{\rmdefault}{ptm}
\geometry{a4paper,centering,scale=0.8}
\title{\heiti 《岩石物理学》结课作业}
\usepackage{txfonts}
\usepackage{multirow,makecell}
\usepackage{fancyvrb}
\usepackage{pifont}
\bibliographystyle{unsrt}%文献
\begin{document}
\zihao{4}
\begin{titlepage}
\vspace*{40mm}
\begin{center}
{\heiti\Huge 《岩石物理学》结课作业}\\[30mm]%有封面时使用
{\Large 王亮}\\[5mm]%有封面时使用
地质资源与地质工程\\\texttt{学号:102014079}\\[80mm]%有封面时使用
2015年6月27日%此处填写日期%有封面时使用
\end{center}
\end{titlepage} 
\newpage
\thispagestyle{empty}
\mbox{} 
\newpage
\pagenumbering{arabic}
\tableofcontents
\newpage
\begin{abstract}
\pagenumbering{arabic}
这是我的《岩石物理学》课程的结课作业。这份结课作业是学习孙建国编《岩石物理学基础》和席道瑛、徐松林编著的《岩石物理学基础》两本书写成的。
\end{abstract}
\section{绪论}
\subsection{学科意义}
岩石是由矿物或类似矿物的物质组成的固体集合体,是地球表层和地球内部物质的基本组成部分。
岩石物性是地球物理理论的基础,是储层综合地球物理技术的基础,在地球物理勘探中广泛应用。

岩石是极具复杂性和分散性的地球介质,岩石物理学就是要从实验和理论上研究岩石的物理力学特性,搞清弹性参数与岩石其他物性参数和状态参数之间的关系。
搞清这些关系对岩石弹性波和电磁波传播的影响,以及他们在地球物理和岩石物理数据中的响应不仅是地球物理资料的正反演计算和综合解释所必须的,
而且对深部构造研究、
区域性油气预测、储层预测和油藏描述也是必不可少的。

在促进这种研究的过程中,岩石物理学的任务在于通过实验室岩石物理模拟,建立寻找地球物理参数和岩石物理性质之间的关系。
岩石物理学是地震数据和储层性质的桥梁。
\subsection{研究内容}
岩石有很多物理性质,比如密度、弹性、导电性、导磁性等。这些性质中,有些可以形成可观测的物理场,有些则不能。
岩石物理学主要研究这些能形成物理场的岩石物理性质。
岩石物理学的基本任务是找出岩石的物理参数与岩石结构、组成成分之间的关系和规律。具体地讲,岩石物理学的任务是:
\begin{enumerate}
\item 通过现场和实验室观测确定岩石的物理性质和有关岩石物理参数的具体数值;
\item 找出岩石的物理性质和地质、矿床、工程与工艺参数之间的关系;
\item 为实测地球物理资料的解释工作提供基础数据。
\end{enumerate}
\subsection{研究方法和主要困难}
岩石物理学的研究方法是观察、实验、归纳和总结。

观察是指对岩石中的物理现象的观察,既包括在野外的现场观察,也包括在室内的实验观察。由于想观察地球内部只能靠大陆深钻,而大区域内岩石物理性质有横向变化,所以一孔之见难以说明完全的情况,更不用说打钻的深度。
\section{岩石的密度}
在岩石的密度这一章,我主要学习了两方面的内容:影响密度的因素和密度的测定。
影响密度的主要因素是矿物和孔隙充填物的物质种类和环境的影响。而密度的测定方法有两种即密度的实验室测定和野外测定
\subsection{矿物和孔隙充填物的影响}
首先矿物和孔隙充填物的物质种类是对密度影响的绝对因素。但,其他一些现象或者说条件也会影响岩石的密度,比如类质同象。
而在不同的条件下,孔隙充填物的密度往往有较大变化比如矿化度对充填液体密度的影响。也有些现象影响较小,比如温度。
\subsection{不同岩类的差异和其它一些规律}
沉积岩的密度的范围是$1.2-3.0g/cm^{3}$。沉积岩的沉积作用主要分为压实和胶结两个阶段。成岩后,岩石将在地质静压力下经受长期的破坏作用。
具体表现在岩石的孔隙度逐渐缩小,而密度逐渐增大。构造作用也影响沉积岩的密度。在褶皱区,沉积岩的密度会剧烈变化。

岩浆岩的密度的范围是$2.6-3.5g/cm^{3}$。一般来看,侵入岩和喷出岩的密度有着一定的差别。对于喷出岩,即使在矿物成分相近时,其密度的变化范围也较大。
这主要是喷出岩的原始构造、结构和成岩作用不同而造成的。

变质岩的密度的范围是$2.4-3.1g/cm^{3}$。由于岩石在变质改造过程中经受了一系列的物理化学变化,从而变质岩的密度有可能与原岩有很大不同。
首先,化学结构的变化使岩石的密度发生变化。各种变质岩之间的矿物成分差别很大,即使是同名的岩石组,岩石的密度也会产生很大变化。
在有些条件下,从原岩到全变质岩之间的岩石序列的密度是渐变的。
\subsection{环境的影响}
环境对密度的影响主要体现在随着深度的加深,压强和温度增加到很大对岩石的影响。压强越大,密度越大。
\subsection{密度的实验室测定和野外测定}
实验室测定主要有静水压法和密度仪法。而在野外,岩石密度的测量可以通过下列途径完成:
\begin{enumerate}
\item 地面重力测量;
\item 井中重力测量;
\item 地球物理测井
\item 利用地震波速度。
\end{enumerate}
\subsection{密度模型}
另外,需要指出的一点是。本章的学习除了了解一般的岩石密度的规律外,我认为应该重视岩石密度的模型。
众所周知,地球物理学研究的都是模型。我们在实际科研中需要使用到密度时,其实往往是要用地球的一个密度模型,包含密度的地球模型很多,比如AK135等等。
在我们所用的教材中介绍了一些模型,绝大多数都是拟合方法得到的。而我们平素用的模型则往往利用的地球内部地震波传播的速度。
\section{岩石磁学}
地球的磁场可能是地球最奇妙的物理场了。地球磁场的磁感应强度其实非常小,其磁性不强于一块能吸在冰箱门上的磁铁。
但是如果没有这个弱而遍布全球的磁场,地球的生态系统都将受到宇宙射线的威胁。而地球的磁场来源至今依然是一个迷。
目前,最好的地球物理模型只能模拟出一次倒转,而实际的地磁场却已经发生了多次倒转。
\subsection{岩石的磁性}
岩石由矿物组成,其磁性主要由岩石所含的铁磁性矿物产生,它们在岩石中占的比例也许很少,但对岩石磁性的影响却是不可忽视的。

实验证明,大部分矿物呈顺磁性或反磁性。铁磁性及亚铁磁性矿物主要以铁和钛的氧化物的形式出现,少量的以硫化物的形式出现。
铁磁性矿物主要有铁磁矿、雌黄铁矿、磁赤铁矿、镍磁铁矿、镁铁矿及钛铁矿等。
岩石的磁性也并非只由单个的矿物来确定,而且还依赖于他们的产出方式和矿物的粒度。
\subsubsection{沉积岩的磁性}
沉积岩的磁性及其铁磁性副矿物的含量及成分有关。由于沉积岩中的主要造岩矿物都具有反磁性或弱磁性,所以他们对岩石的磁性几乎没有贡献。
粘土矿物、菱铁矿、黄铁矿、钛铁矿、黑云母等矿物具有强顺磁性,对沉积岩的磁化率有一定贡献。

沉积岩的磁性具有如下特点:
\begin{enumerate}
\item 具有顺磁性和反磁性;
\item 磁性不如火成岩;
\item 碎屑岩磁性较高,通常含砂多的岩石磁性高;
\item 化学和生物沉积岩呈微弱的顺磁性和反磁性。
\end{enumerate}
\subsubsection{火成岩的磁性}
火成岩的磁性有三个特点:
\begin{enumerate}
\item 磁化率由酸性到基性增高;
\item 存在剩磁现象;
\item 磁化率的范围很大;
\end{enumerate}
另外,火成岩的磁性与其类型有关。主要的侵入岩都属于弱碱性的范围,喷出岩的磁性较强。
\subsubsection{变质岩的磁性}
在三大岩类中,变质岩的磁化率和天然剩余磁化强度的变化范围最大。既有很强铁磁性的变质岩岩体,也可以见到具有反磁性的变质岩岩体。
变质岩的磁性受原岩的磁性和变质过程有关,如果原岩是沉积岩,则磁性较弱,但如果原岩是具有强磁性的岩石,则变质后的岩石也具有很强的磁性。
依据变质岩的磁性可以将变质岩分为两类:铁磁\cndash{}顺磁性和铁磁性。

下面讨论不同变质作用对变质岩磁性的影响。

低相条件下的岩石经过区域变质作用后产生的维结晶板岩以铁磁\cndash{}顺磁性岩石变种为主。
天然剩余磁化强度或者不存在或者很小。在区域变质作用更高相的条件下,生成不同成分的片麻岩、角闪岩和其他岩石。
与板岩相比,这些岩石具有较强磁性。超变质作用能引起岩石磁性的复杂变化。
在掩饰的花岗岩化和硅碱交代作用下,含铁的硅酸盐发生分解,生成磁铁矿,致使岩石的磁化率升高。
在动力变质作用和接触变质作用下生成的岩石具有不固定的磁性。这和原岩有关,又和变质时的温度有关。
自变质和热液交代作用将使岩石的磁化率在大范围内发生变化。
\subsection{岩石磁性的野外和实验室测量}
岩矿石磁性参数测定是岩石磁学研究的基础性工作。
按场地的不同,可以分为野外现场和实验室内测定两大类。
在勘探地球物理中,必须通过实验测定的磁性参数是总磁化强度、剩余磁化强度及磁化率。
\\subsubsection{岩石磁性的实验室测定}
在实验室内可以测定岩石磁化率、剩余磁化强度、饱和磁化强度、矫顽磁力、居里点和磁化强度的稳定性等参数
依据原理的不同,可以分为磁法测量和感应磁法测量
\\subsubsection{岩石磁性的野外测定}
实验室内所测样品是非自然状态下的岩石样品。
因此,在实验室内所得的岩石磁性参数值与其在自然状态下的参数值会有一定的差别。
为了得到岩石在自然条件下的参数,可以采用地面、航空或井中磁测的有关方法技术。
在野外观测数据的基础上,通过利用现代地球物理反演方法,可以计算出地下磁性体的总磁化强度和磁化率。
\subsection{地磁场的历史}
岩石有可能把从前地球磁场的情况作为化石而保存下来,这就是天然岩石所具有的永久磁化性质。
岩浆从地下喷出,凝固后形成火山岩。
在测定玄武岩等火山岩的磁性时,常常发现它们具有意想不到的强磁性。
虽然它的强度只是真正磁铁的数千分之一,但仪器足够测出其磁性方向。
火山岩最初以灼热流体形式喷发至地表,它的温度比居里点高。
但当它冷凝经过居里点时,顺着地球磁场的方向呈永久磁石状态,在岩石中就以化石化的方式把这种方向保存下来。
\section{岩石电学}
岩石电性中,对地球物理最有意义的是电阻率、介电常数、自然极化和激发极化特性以及压电效应等参数。
由电磁感应原理,知电性的变化与地磁变化有密切的关系。
电磁仪器测量的频率范围在$10^{-6}-10^{11}$,探测深度从几分之一米到几千米的深度。
它的穿透性与变化类型是由物质的电磁性决定的。
一般来说,地球上多数物质磁性与自由空间的差别可以忽略不计,因此,电性是决定穿透深度的最重要物理量。
地下电流主要是变化磁场直接感应的。这种电流与感应磁场强度、地球结构和岩矿石的电学性质等因素有关系。
电性对各种参数十分敏感。
这种特殊的敏感性可用于测定和推断地球内部岩层的变化情况,同时这种灵敏性造成了变化因素的多解性和实验室电性研究的复杂性。
\subsection{岩矿石的电阻率}
矿物的电阻率变化范围很大,可达24个数量级,从良导体到绝缘体都有。

岩石的电阻率由导电矿物的相对含量和岩石所含的流体及其分布来决定。
按导电特性岩石可以分为两类没有离子导体水极化的岩石和含有离子导体及有强烈的孔隙水计划效应的岩石。
岩石的电阻率变化范围也很大。在天然状态下的岩石的电阻率与很多因素有关,其一就是岩性。
对于火成岩和变质岩,其导电性主要取决于岩石的含水量。而对于沉积岩,特点是离子导电。
因为沉积岩的含水量主要由层间地下水决定,所以地下水的矿化度、动态和水文化学特点对岩石的电阻率有很大影响
\subsection{岩石的极化}
岩石的极化分为自然极化和激发极化。
\subsubsection{岩石的自然极化}
岩石的自然极化是是岩石的一种重要的电学现象:在外电场不存在时,在测量电极上依然观测到一定数量的电位差。
岩石的自然极化现象主要与岩石中发生的动电效应和电化学过程有关。
\subsubsection{岩石的激发极化}
电流在介入和断开时,会产生随时间变化的附加电场的现象叫做激发极化。
岩石的激发极化效应得到了广泛的应用。
到目前为止,激发极化法依旧是找寻硫化物矿床的有效方法,在寻找地下水和进行环境与工程勘察方面也是效果良好的手段之一。

影响岩矿石激发极化效应的主要因素是围岩电阻率、矿物成分和矿物颗粒大小、底层水的矿化度、阳离子交换律和孔隙度。
\section{岩石声学}
岩石声学利用连续介质力学的一般规律研究声波在岩石中的产生、传播、接受及沈波在其传播过程中与岩石之间的相互作用。
它是声学的一个分支、是固体地球物理学和勘探地球物理学的重要组成部分,是地震学和勘探地震学以及声波测井技术的物理基础和资料解释依据。
\subsection{弹性参数及其模型}
\subsubsection{弹性参数}
拉梅常数$\lambda$、$\mu$用于理论研究。在工程领域一般用下列参数:
\begin{enumerate}
\item $E$:杨氏模量(描述轴或杆的轴向形变)
$$E = \frac {\sigma_{xx} } {\epsilon_{xx}}$$
\item $v$:泊松比(横向及纵向的形变比)
$$v=-\frac{\epsilon_{yy}}{\epsilon_{xx}}$$
\item $K$:体积模量(静水压力和体积变化之比),即
$$K=-V\frac{\partial p}{\partial V}$$
\item $\alpha$:可压缩量(体积模量的倒数)
$$\alpha=\frac{1}{K}$$
\end{enumerate}
\subsubsection{孔隙流体的弹性参数}
孔隙流体的弹性参数依赖于地层的温度、压力和流体的种类。
由于流体对于剪切应力没有反抗作用,所以流体的弹性参数只有体积模量$K$。
\subsubsection{矿物的弹性参数}
矿物的弹性主要取决于构成矿物的元素、成分、亲和方式及其原子结构。
由于原子核的密度很高,所以原子核的弹性可以忽略不计。
与此相反,原子的电子壳层具有一定的结构,在外力的作用下会出现反抗性变的内力。所以,矿物的弹性来源于组成元素的电子壳层。
\subsubsection{岩石的弹性参数}
岩石是矿物和空隙充填物的堆积混合体,具有不连续性、不均匀性和各向异性。所以,将岩石看成是完全弹性体是有条件的。
当外力的作用很小、作用时间很短时,岩石接近于弹性体。在远离震源的观测点处,地震波产生的力符合这样的要求。
所以,在进行地震波研究中利用完全弹性介质的假设具有合理性。
但是,在声波测井中,由于所用的频率很高,有流体和固体间的相互作用所引起的现象不能忽略,必须对完全弹性的假设进行必要修正。

岩石的弹性与岩石的孔隙度密切有关。虽然许多结晶岩的孔隙度很小,但对岩石的弹性参数影响却很大。
例如,$1\%$孔隙度的花岗岩在其空隙压实前后的弹性参数可相差5倍之多。
\subsubsection{岩石的弹性参数模型}
岩石的弹性参数除了和岩性有关,还与岩石的结构有关。
虽然,岩石在微观上或在局部范围内是不均匀的,但是它们对于频率足够低的声波的响应确实可以等效为某种均匀弹性参数。
常见的模型有:层状介质、离散颗粒堆积介质等
\subsection{影响岩石速度的主要因素}
实验证明,岩石的地震波速度主要与下列因素有关
\begin{enumerate}
\item 岩性

岩性是影响岩石速度的一个重要因素,但不同的岩性,速度值可能相同,所以不能用速度值来区别岩性。
尽管如此,地震波的速递与岩性还是有一定的关系。
例如,速度高的沉积岩可能是砂岩或者泥岩。
\item 密度

密度对速度的影响直接反映在速度的定义公式中:
$$v_{p}=\sqrt{\frac{\lambda+2\mu}{\rho}}$$
$$v_{p}=\sqrt{\frac{\mu}{\rho}}$$
1974年,Grandner等人发现了Grandner公式:
$$\rho=0.31v_{p}^{\frac{1}{4}}$$
\item 孔隙度

孔隙度,在最简单的条件下,对速度的影响由时间平均方程给出:
$$\frac{1}{v}=\frac{\phi}{v_{p}}+\frac{(1-\phi)}{v_{m}}$$
\item 压力和埋深

当压力增加时,岩石中的空隙和裂隙被证实,因为速度增加。如果岩石的孔隙度和裂隙度均接近于零,则速度与压力无关。
\item 含水饱和度

速度和岩石的含税饱和度有一定关系。当含水饱和度为零(干燥岩石)和1(百分百含水)时,速度明显增大。
如果岩石的含水饱和度位于$10\%$和$90\%$之间时,速度和含水饱和度之间的关系比较弱。
\item 空隙流体粘度

空隙流体粘度对速度有很大影响。一般来讲,速度随着黏度的增大而增大。
\item 地质年代

地质年代越老,速度值越高。
这是因为年代老的岩石受构造应力和胶结作用的时间更长,具有更小的孔隙度。
\item 温度

温度每增加100摄氏度,速度会减小$5\%-6\%$。
当岩石的孔隙流体含有重质原油和焦油时,温度对速度的影响比较大。
当岩石的温度低于冰点时,水饱和岩石的速度会有明显的提高。
\item 频率

速度与频率有关。随着频率增高,速度会加大。
实验证明,当频率在HZ级到MHZ级变化时,频散现象很弱。
这意味着,在地面地震勘探中,可以忽略频散。
\end{enumerate}
\subsection{岩石对声波能量的吸收}
声波能量在岩石中传播时会不断减弱。
除了由于几何扩散引起的能量衰减,还有能量由弹性能转化为热能,这种声波能量减弱的过程就是吸收。
也就是说,吸收式弹性势能转换成热能的过程。

由于吸收现象的存在,岩石中的声能量将会随着传播距离的增加而逐渐减弱,直到最后消失。
在机制上,认为下列因素会引起吸收:
\begin{enumerate}
\item 孔隙流体的相对运动;
\item 介质间的摩擦;
\item 空隙和流体之间的接触面的相对运动;
\item 大孔隙岩石中的喷流效应和局部饱和效应;
\item 几何漫射和薄层效应。
\end{enumerate}
\subsection{岩石声波速度的测量方法}
\subsubsection{实验室测量}
在实验室内进行岩石声速测量需要特殊的测量系统和特殊的标本形状,所用频率一般在$10^{6}HZ$以上。
因此,实验室内测量得到的实际是超声波在岩石中的速率。

实验室内观测的主要方法是行波法。
通过特殊的换能器,首先将电磁振荡能量转化为声波能量,然后测量声波通过标本的时间。
\subsubsection{野外现场测量}
在野外,常用的速度测量方法是地震测井、垂直地震剖面和常规及全波列声波测井。
也可以根据几何地震学中有关公式直接从反射地震资料中提取速度信息。
\section{岩石热学}
\subsection{矿物和岩石的导热性质}
矿物的导热率与其结构、化学成分、密度、弹性模量和矿物所处的环境有关。
在造岩矿物中,石英的热导率较高,而云母较低。金属矿物具有较高热导率。
由于晶格构造的原因,矿物的热导率具有各向异性。

岩石的热力学特点主要取决于岩石的结构构造和其中所含的造岩矿物及胶结物的热学性质。
对于致密的岩浆岩和变质岩,其导热性与矿物成分、内部构造、温度和压力有关。
如果岩石中存在裂隙,则其导热性与压力、裂隙的性质、几何形状有关。
\subsection{岩石热导率和温度的测定}
\subsubsection{实验室测量}
在实验室内测量热导率的途径有两种:基于热流测定和基于热量测定。

再给予测定热流的方法中,按热流是否随时间变化分为恒定热流法和非恒定热流法两种。
恒定热流法精度较高,但所需时间很长,仪器笨重,边界条件难以重复实现。非恒定热流法需要专用实验设备。
\subsubsection{钻孔测量}
在钻孔中测量温度的方法称为井温测井,是研究区域地热场的直接方法。
根据温度沿井孔的变化,可以推断出在钻孔前地层内的原始温度场。
\subsubsection{大地热流密度测定}
大地热流密度是地温梯度和热导率的乘积。大地热流密度测量可以在陆地也可以在海洋实施。
在陆地上,为了避开各种干扰,一般在钻孔中进行地温测量,在实验室内测定热导率。
在海洋,一般在深海区($2000m$以下)进行温度测量。
由于在深海区大范围内的温度基本不遂时间的改变而改变,所以海底表面的热流密度就代表了地球内部的热流密度。
\nocite{shiye}
\nocite{quanjing}
\bibliography{math}
\end{document}